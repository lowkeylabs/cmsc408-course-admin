% Options for packages loaded elsewhere
\PassOptionsToPackage{unicode}{hyperref}
\PassOptionsToPackage{hyphens}{url}
\PassOptionsToPackage{dvipsnames,svgnames,x11names}{xcolor}
%
\documentclass[
  letterpaper,
  DIV=11,
  numbers=noendperiod]{scrartcl}

\usepackage{amsmath,amssymb}
\usepackage{iftex}
\ifPDFTeX
  \usepackage[T1]{fontenc}
  \usepackage[utf8]{inputenc}
  \usepackage{textcomp} % provide euro and other symbols
\else % if luatex or xetex
  \usepackage{unicode-math}
  \defaultfontfeatures{Scale=MatchLowercase}
  \defaultfontfeatures[\rmfamily]{Ligatures=TeX,Scale=1}
\fi
\usepackage{lmodern}
\ifPDFTeX\else  
    % xetex/luatex font selection
\fi
% Use upquote if available, for straight quotes in verbatim environments
\IfFileExists{upquote.sty}{\usepackage{upquote}}{}
\IfFileExists{microtype.sty}{% use microtype if available
  \usepackage[]{microtype}
  \UseMicrotypeSet[protrusion]{basicmath} % disable protrusion for tt fonts
}{}
\makeatletter
\@ifundefined{KOMAClassName}{% if non-KOMA class
  \IfFileExists{parskip.sty}{%
    \usepackage{parskip}
  }{% else
    \setlength{\parindent}{0pt}
    \setlength{\parskip}{6pt plus 2pt minus 1pt}}
}{% if KOMA class
  \KOMAoptions{parskip=half}}
\makeatother
\usepackage{xcolor}
\setlength{\emergencystretch}{3em} % prevent overfull lines
\setcounter{secnumdepth}{5}
% Make \paragraph and \subparagraph free-standing
\ifx\paragraph\undefined\else
  \let\oldparagraph\paragraph
  \renewcommand{\paragraph}[1]{\oldparagraph{#1}\mbox{}}
\fi
\ifx\subparagraph\undefined\else
  \let\oldsubparagraph\subparagraph
  \renewcommand{\subparagraph}[1]{\oldsubparagraph{#1}\mbox{}}
\fi


\providecommand{\tightlist}{%
  \setlength{\itemsep}{0pt}\setlength{\parskip}{0pt}}\usepackage{longtable,booktabs,array}
\usepackage{calc} % for calculating minipage widths
% Correct order of tables after \paragraph or \subparagraph
\usepackage{etoolbox}
\makeatletter
\patchcmd\longtable{\par}{\if@noskipsec\mbox{}\fi\par}{}{}
\makeatother
% Allow footnotes in longtable head/foot
\IfFileExists{footnotehyper.sty}{\usepackage{footnotehyper}}{\usepackage{footnote}}
\makesavenoteenv{longtable}
\usepackage{graphicx}
\makeatletter
\def\maxwidth{\ifdim\Gin@nat@width>\linewidth\linewidth\else\Gin@nat@width\fi}
\def\maxheight{\ifdim\Gin@nat@height>\textheight\textheight\else\Gin@nat@height\fi}
\makeatother
% Scale images if necessary, so that they will not overflow the page
% margins by default, and it is still possible to overwrite the defaults
% using explicit options in \includegraphics[width, height, ...]{}
\setkeys{Gin}{width=\maxwidth,height=\maxheight,keepaspectratio}
% Set default figure placement to htbp
\makeatletter
\def\fps@figure{htbp}
\makeatother

\KOMAoption{captions}{tableheading}
\makeatletter
\makeatother
\makeatletter
\makeatother
\makeatletter
\@ifpackageloaded{caption}{}{\usepackage{caption}}
\AtBeginDocument{%
\ifdefined\contentsname
  \renewcommand*\contentsname{Table of contents}
\else
  \newcommand\contentsname{Table of contents}
\fi
\ifdefined\listfigurename
  \renewcommand*\listfigurename{List of Figures}
\else
  \newcommand\listfigurename{List of Figures}
\fi
\ifdefined\listtablename
  \renewcommand*\listtablename{List of Tables}
\else
  \newcommand\listtablename{List of Tables}
\fi
\ifdefined\figurename
  \renewcommand*\figurename{Figure}
\else
  \newcommand\figurename{Figure}
\fi
\ifdefined\tablename
  \renewcommand*\tablename{Table}
\else
  \newcommand\tablename{Table}
\fi
}
\@ifpackageloaded{float}{}{\usepackage{float}}
\floatstyle{ruled}
\@ifundefined{c@chapter}{\newfloat{codelisting}{h}{lop}}{\newfloat{codelisting}{h}{lop}[chapter]}
\floatname{codelisting}{Listing}
\newcommand*\listoflistings{\listof{codelisting}{List of Listings}}
\makeatother
\makeatletter
\@ifpackageloaded{caption}{}{\usepackage{caption}}
\@ifpackageloaded{subcaption}{}{\usepackage{subcaption}}
\makeatother
\makeatletter
\@ifpackageloaded{tcolorbox}{}{\usepackage[skins,breakable]{tcolorbox}}
\makeatother
\makeatletter
\@ifundefined{shadecolor}{\definecolor{shadecolor}{rgb}{.97, .97, .97}}
\makeatother
\makeatletter
\makeatother
\makeatletter
\makeatother
\ifLuaTeX
  \usepackage{selnolig}  % disable illegal ligatures
\fi
\IfFileExists{bookmark.sty}{\usepackage{bookmark}}{\usepackage{hyperref}}
\IfFileExists{xurl.sty}{\usepackage{xurl}}{} % add URL line breaks if available
\urlstyle{same} % disable monospaced font for URLs
\hypersetup{
  pdftitle={CMSC 508 Outcomes and Indicators},
  colorlinks=true,
  linkcolor={blue},
  filecolor={Maroon},
  citecolor={Blue},
  urlcolor={Blue},
  pdfcreator={LaTeX via pandoc}}

\title{CMSC 508 Outcomes and Indicators}
\author{}
\date{}

\begin{document}
\maketitle
\ifdefined\Shaded\renewenvironment{Shaded}{\begin{tcolorbox}[boxrule=0pt, borderline west={3pt}{0pt}{shadecolor}, sharp corners, breakable, interior hidden, frame hidden, enhanced]}{\end{tcolorbox}}\fi

\renewcommand*\contentsname{Table of contents}
{
\hypersetup{linkcolor=}
\setcounter{tocdepth}{3}
\tableofcontents
}
\hypertarget{student-learning-outcome-1}{%
\section{Student Learning Outcome 1}\label{student-learning-outcome-1}}

``Analyze a complex computing problem and apply principles of computing
and other relevant disciplines to identify solutions.''

\begin{longtable}[]{@{}
  >{\raggedright\arraybackslash}p{(\columnwidth - 8\tabcolsep) * \real{0.2203}}
  >{\raggedright\arraybackslash}p{(\columnwidth - 8\tabcolsep) * \real{0.2000}}
  >{\raggedright\arraybackslash}p{(\columnwidth - 8\tabcolsep) * \real{0.1966}}
  >{\raggedright\arraybackslash}p{(\columnwidth - 8\tabcolsep) * \real{0.1898}}
  >{\raggedright\arraybackslash}p{(\columnwidth - 8\tabcolsep) * \real{0.1932}}@{}}
\toprule\noalign{}
\begin{minipage}[b]{\linewidth}\raggedright
Performance Indicator
\end{minipage} & \begin{minipage}[b]{\linewidth}\raggedright
Exceeds Expectations
\end{minipage} & \begin{minipage}[b]{\linewidth}\raggedright
Meets Expectations
\end{minipage} & \begin{minipage}[b]{\linewidth}\raggedright
Developing
\end{minipage} & \begin{minipage}[b]{\linewidth}\raggedright
Unsatisfactory
\end{minipage} \\
\midrule\noalign{}
\endhead
\bottomrule\noalign{}
\endlastfoot
\textbf{1. Problem Analysis} & Effectively deconstructs complex database
problems into their constituent elements, demonstrating a deep
understanding of relevant concepts and relationships. & Accurately
breaks down complex database problems, considering key concepts and
relationships. & Attempts to deconstruct database problems but overlooks
some critical aspects. & Struggles to break down database problems,
missing important concepts. \\
\textbf{2. Interdisciplinary Application} & Skillfully applies
principles from database theory and other relevant disciplines to
propose innovative solutions that address the problem's complexity. &
Appropriately integrates database theory with other disciplines to
suggest feasible solutions. & Attempts to combine database principles
with other disciplines, with limited effectiveness. & Fails to connect
database principles with other disciplines, leading to ineffective
solutions. \\
\textbf{3. Solution Identification} & Expertly identifies multiple
potential solutions, considering their implications and trade-offs
thoroughly. & Identifies suitable solutions and discusses their
potential impacts and trade-offs. & Presents limited solution options
with incomplete insight into their implications. & Offers single,
inadequate solution choice without proper consideration of its
implications. \\
\end{longtable}

\hypertarget{student-learning-outcome-2}{%
\section{Student Learning Outcome 2}\label{student-learning-outcome-2}}

``Design, implement and evaluate a computing-based solution to meet a
given set of computing requirements in the context of the discipline of
computer science.''

\begin{longtable}[]{@{}
  >{\raggedright\arraybackslash}p{(\columnwidth - 8\tabcolsep) * \real{0.3145}}
  >{\raggedright\arraybackslash}p{(\columnwidth - 8\tabcolsep) * \real{0.1855}}
  >{\raggedright\arraybackslash}p{(\columnwidth - 8\tabcolsep) * \real{0.1792}}
  >{\raggedright\arraybackslash}p{(\columnwidth - 8\tabcolsep) * \real{0.1572}}
  >{\raggedright\arraybackslash}p{(\columnwidth - 8\tabcolsep) * \real{0.1635}}@{}}
\toprule\noalign{}
\begin{minipage}[b]{\linewidth}\raggedright
Performance Indicator
\end{minipage} & \begin{minipage}[b]{\linewidth}\raggedright
Exceeds Expectations
\end{minipage} & \begin{minipage}[b]{\linewidth}\raggedright
Meets Expectations
\end{minipage} & \begin{minipage}[b]{\linewidth}\raggedright
Developing
\end{minipage} & \begin{minipage}[b]{\linewidth}\raggedright
Unsatisfactory
\end{minipage} \\
\midrule\noalign{}
\endhead
\bottomrule\noalign{}
\endlastfoot
\textbf{1. Solution Design} & Develops a comprehensive and innovative
solution plan, & Creates a well-structured solution plan with & Creates
a solution plan with some gaps and & Fails to create a coherent solution
plan \\
& incorporating advanced techniques and considerations. & consideration
of key design principles. & inconsistencies in design choices. & or
includes inappropriate design elements. \\
\textbf{2. Implementation Proficiency} & Implements the solution with
exceptional proficiency, & Successfully implements the solution
according & Implements the solution with minor errors & Struggles to
implement the solution, resulting \\
& demonstrating mastery of relevant tools and practices. & to plan using
appropriate tools and practices. & but faces significant challenges in
execution. & in a solution that does not meet requirements. \\
\textbf{3. Evaluation and Reflection} & Conducts a thorough evaluation,
comparing the & Performs a comprehensive evaluation, analyzing &
Conducts a basic evaluation, but lacks depth & Provides minimal or no
evaluation of the \\
& solution against alternatives, and showcasing insights. & strengths
and weaknesses of the solution. & in insights and critical analysis. &
solution's effectiveness and shortcomings. \\
\end{longtable}

Please note that these descriptions are concise and fit within a single
cell of the table. You can adapt and modify them according to the
specific context of your assessment and the level of students'
expertise.

\hypertarget{student-learning-outcome-3}{%
\section{Student Learning Outcome 3}\label{student-learning-outcome-3}}

``Communicate effectively in a variety of professional contexts;''

\begin{longtable}[]{@{}
  >{\raggedright\arraybackslash}p{(\columnwidth - 8\tabcolsep) * \real{0.2473}}
  >{\raggedright\arraybackslash}p{(\columnwidth - 8\tabcolsep) * \real{0.2366}}
  >{\raggedright\arraybackslash}p{(\columnwidth - 8\tabcolsep) * \real{0.2151}}
  >{\raggedright\arraybackslash}p{(\columnwidth - 8\tabcolsep) * \real{0.1290}}
  >{\raggedright\arraybackslash}p{(\columnwidth - 8\tabcolsep) * \real{0.1720}}@{}}
\toprule\noalign{}
\begin{minipage}[b]{\linewidth}\raggedright
Performance Indicator
\end{minipage} & \begin{minipage}[b]{\linewidth}\raggedright
Exceeds Expectations
\end{minipage} & \begin{minipage}[b]{\linewidth}\raggedright
Meets Expectations
\end{minipage} & \begin{minipage}[b]{\linewidth}\raggedright
Developing
\end{minipage} & \begin{minipage}[b]{\linewidth}\raggedright
Unsatisfactory
\end{minipage} \\
\midrule\noalign{}
\endhead
\bottomrule\noalign{}
\endlastfoot
\textbf{1. Explanation Clarity} & Can clearly and concisely explain
complex concepts using various professional contexts, adapting language
and examples to audience needs. & Can effectively explain concepts in
professional contexts, maintaining audience engagement and clarity. &
Explains concepts with minor inaccuracies or confusion, and requires
guidance to understand key concepts. & Struggles to explain basic
concepts and lacks coherence; unable to convey key points. \\
\textbf{2. Visual Aids Utilization} & Skillfully employs visual aids
(e.g., diagrams, charts) to enhance understanding of concepts in
presentations and technical documents. & Effectively uses visual aids to
support explanations, enhancing comprehension of topics and improving
the overall professionalism. & Attempts to use visual aids but they lack
clarity or relevance, and they may not align with the content being
presented. & Rarely uses visual aids or uses them inappropriately,
failing to enhance understanding or professionalism. \\
\textbf{3. Technical Writing Proficiency} & Demonstrates excellent
technical writing skills in various formats (reports, documentation),
effectively conveying concepts and solutions. & Writes well-structured
technical documents, maintaining technical accuracy and clarity. &
Presents ideas in a disorganized or unclear manner, making it difficult
to follow the logical flow of thoughts or arguments. & Struggles to
convey ideas coherently in written form, lacking appropriate terminology
and logical structure. \\
\textbf{4. Dialog With Peers and Collegues} & Skillfully communicates
non-technical and technical ideas in clear, thoughtful and meaningful
ways through written and/or oral communication. & Effectively
communicates non-technical and technical ideas in thoughtful and
meaningful ways through written and/or oral communication. &
Communicates ideas in a disorganized or unclear manner, making it
difficult to follow the logical flow of thoughts or arguments. &
Struggles to convey ideas coherently in written form, lacking
appropriate terminology and logical structure. \\
\end{longtable}

\hypertarget{student-learning-outcome-4}{%
\section{Student Learning Outcome 4}\label{student-learning-outcome-4}}

``Recognize professional responsibilities and make informed judgments in
computing practice based on legal and ethical principles;''

\begin{longtable}[]{@{}
  >{\raggedright\arraybackslash}p{(\columnwidth - 8\tabcolsep) * \real{0.2358}}
  >{\raggedright\arraybackslash}p{(\columnwidth - 8\tabcolsep) * \real{0.2033}}
  >{\raggedright\arraybackslash}p{(\columnwidth - 8\tabcolsep) * \real{0.1870}}
  >{\raggedright\arraybackslash}p{(\columnwidth - 8\tabcolsep) * \real{0.1870}}
  >{\raggedright\arraybackslash}p{(\columnwidth - 8\tabcolsep) * \real{0.1870}}@{}}
\toprule\noalign{}
\begin{minipage}[b]{\linewidth}\raggedright
Performance Indicator
\end{minipage} & \begin{minipage}[b]{\linewidth}\raggedright
Exceeds Expectations
\end{minipage} & \begin{minipage}[b]{\linewidth}\raggedright
Meets Expectations
\end{minipage} & \begin{minipage}[b]{\linewidth}\raggedright
Developing
\end{minipage} & \begin{minipage}[b]{\linewidth}\raggedright
Unsatisfactory
\end{minipage} \\
\midrule\noalign{}
\endhead
\bottomrule\noalign{}
\endlastfoot
\textbf{1. Ethical Decision-Making} & Consistently applies legal and
ethical principles to complex scenarios, demonstrating deep
understanding of implications in database theory and design. & Applies
legal and ethical principles to various scenarios, showing understanding
of implications in database theory and design. & Demonstrates some
awareness of legal and ethical considerations, but struggles to apply
them consistently in database-related contexts. & Lacks awareness of
legal and ethical principles in various scenarios. \\
\textbf{2. Professional Responsibility} & Takes a proactive role in
identifying potential legal and ethical issues in database projects,
consistently integrating them into project planning. & Identifies
potential legal and ethical considerations in database projects,
considering their impact on project planning and execution. & Recognizes
basic legal and ethical considerations, but may overlook critical
aspects, potentially leading to oversights and risks. & Fails to
recognize or address legal and ethical considerations. \\
\textbf{3. Ethical Dilemma Resolution} & Effectively navigates complex
ethical dilemmas in database work, balancing technical requirements with
legal and moral considerations for optimal outcomes. & Analyzes and
addresses common ethical dilemmas in database work, making balanced
decisions based on legal and moral considerations. & Recognizes
straightforward ethical dilemmas but may struggle with more complex
situations or defaults to technical aspects over ethical considerations.
& Struggles to identify ethical dilemmas and lacks awareness of legal
and moral considerations. \\
\end{longtable}

\hypertarget{student-learning-outcome-5}{%
\section{Student Learning Outcome 5}\label{student-learning-outcome-5}}

``Function effectively as a member or leader of a team engaged in
activities appropriate to the program's discipline;''

\begin{longtable}[]{@{}
  >{\raggedright\arraybackslash}p{(\columnwidth - 8\tabcolsep) * \real{0.2419}}
  >{\raggedright\arraybackslash}p{(\columnwidth - 8\tabcolsep) * \real{0.2016}}
  >{\raggedright\arraybackslash}p{(\columnwidth - 8\tabcolsep) * \real{0.1855}}
  >{\raggedright\arraybackslash}p{(\columnwidth - 8\tabcolsep) * \real{0.1855}}
  >{\raggedright\arraybackslash}p{(\columnwidth - 8\tabcolsep) * \real{0.1855}}@{}}
\toprule\noalign{}
\begin{minipage}[b]{\linewidth}\raggedright
Performance Indicator
\end{minipage} & \begin{minipage}[b]{\linewidth}\raggedright
Exceeds Expectations
\end{minipage} & \begin{minipage}[b]{\linewidth}\raggedright
Meets Expectations
\end{minipage} & \begin{minipage}[b]{\linewidth}\raggedright
Developing
\end{minipage} & \begin{minipage}[b]{\linewidth}\raggedright
Unsatisfactory
\end{minipage} \\
\midrule\noalign{}
\endhead
\bottomrule\noalign{}
\endlastfoot
\textbf{1. Collaborative Contribution} & Consistently takes a proactive
role in leading and contributing to team activities related to database
theory and design, demonstrating excellent communication and
problem-solving skills. & Effectively participates in team activities
related to database theory and design, contributing positively through
communication and problem-solving. & Participates in team activities but
occasionally struggles with communication or problem-solving aspects,
requiring some guidance. & Rarely contributes to team activities,
displaying poor communication and problem-solving skills, hindering
progress. \\
\textbf{2. Team Coordination} & Successfully organizes and manages team
tasks for database projects, ensuring efficient resource allocation and
timely goal achievement. & Efficiently participates in coordinating team
tasks for database projects, assisting in resource allocation and goal
achievement. & Occasionally experiences challenges in organizing tasks
or allocating resources within the team, leading to minor delays. &
Struggles to coordinate tasks or manage resources, causing significant
delays and confusion within the team. \\
\textbf{3. Conflict Resolution} & Skillfully mediates conflicts within
the team, fostering a harmonious working environment, and finding
constructive solutions that benefit the project and team dynamics. &
Effectively addresses conflicts within the team, promoting a positive
working environment, and contributing to productive solutions. &
Encounters difficulties in resolving conflicts, occasionally resulting
in unresolved tension that affects teamwork and project progress. &
Struggles to manage conflicts, leading to unresolved issues and creating
a negative atmosphere within the team. \\
\end{longtable}

\hypertarget{student-learning-outcome-6}{%
\section{Student Learning Outcome 6}\label{student-learning-outcome-6}}

``Apply computer science theory and software development fundamentals to
produce computing-based solutions;''

\begin{longtable}[]{@{}
  >{\raggedright\arraybackslash}p{(\columnwidth - 8\tabcolsep) * \real{0.2335}}
  >{\raggedright\arraybackslash}p{(\columnwidth - 8\tabcolsep) * \real{0.1929}}
  >{\raggedright\arraybackslash}p{(\columnwidth - 8\tabcolsep) * \real{0.1878}}
  >{\raggedright\arraybackslash}p{(\columnwidth - 8\tabcolsep) * \real{0.1929}}
  >{\raggedright\arraybackslash}p{(\columnwidth - 8\tabcolsep) * \real{0.1929}}@{}}
\toprule\noalign{}
\begin{minipage}[b]{\linewidth}\raggedright
Performance Indicator
\end{minipage} & \begin{minipage}[b]{\linewidth}\raggedright
Exceeds Expectations
\end{minipage} & \begin{minipage}[b]{\linewidth}\raggedright
Meets Expectations
\end{minipage} & \begin{minipage}[b]{\linewidth}\raggedright
Developing
\end{minipage} & \begin{minipage}[b]{\linewidth}\raggedright
Unsatisfactory
\end{minipage} \\
\midrule\noalign{}
\endhead
\bottomrule\noalign{}
\endlastfoot
\textbf{1: Database Design} & Designs complex databases with optimal
performance and scalability, showcasing expertise in normalization,
indexing, and query optimization. & Designs functional databases with
appropriate structure and functionality, demonstrating understanding of
core principles. & Designs databases with minor issues, requiring
revisions or adjustments after feedback and guidance. & Designs
databases with significant flaws that hinder functionality and
usability. \\
\textbf{2: Query Development} & Develops intricate SQL queries that
manipulate and extract data from databases, demonstrating advanced
understanding of joins, subqueries, and optimization techniques. &
Creates effective SQL queries to retrieve and manipulate data, applying
fundamental principles of querying and data retrieval. & Constructs
basic SQL queries, but lacks complexity or makes errors in syntax and
logic, leading to inaccurate results. & Struggles to write accurate and
functional SQL queries, resulting in incorrect or incomplete results. \\
\textbf{3: Software-Database Integration} & Integrates databases
seamlessly into software projects, ensuring data integrity and security.
Demonstrates the ability to design and implement APIs and communication
protocols between applications and databases. & Connects software
applications with databases effectively, maintaining data consistency
and security. Integrates basic APIs and functionalities for
communication between software and databases. & Attempts to integrate
databases with software, but encounters issues with implementation,
leading to errors or inconsistencies in data handling. & Struggles to
establish any connection between software and databases, resulting in a
failure to interact with data stored. \\
\end{longtable}



\end{document}
